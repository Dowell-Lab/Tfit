\documentclass[11pt]{article}
\usepackage{url}
\usepackage{color}
\usepackage[pdftex]{graphicx}
\usepackage{amsmath}
\usepackage{fancyhdr}
\usepackage[margin=1in,footskip=0.25in]{geometry}
\usepackage{hyperref}

\pagestyle{fancy}
%\pagestyle{empty}

\title{Tfit Notes}
\author{Robin D. Dowell}
\date{}
\lhead{Tfit Design Notes}
\rhead{Dowell}
\date{ }

\begin{document}
\maketitle
\noindent $*$ Corresponding author: robin.dowell@colorado.edu

\section{Coding conventions}
This document assumes that the github Tfit tree has been checked out into a 
directory called "Tfit".   Within the Tfit code base the primary code for the 
algorithm lives in Tfit/src and is written in C++.   Compiling the code is 
done via the Makefile system (cd Tfit/src/; make). 

All methods/classes with logic (i.e. getters and setters generally do not 
have logic, unless they test the inputs) should have a Unit test in 
Tfit/test/src.  Unit tests use the gmock/gtest framework from Google.   
Testing uses cMake to build the appropriate tests (cd Tfit/test/build; 
cmake .. ; make; ./TestTfit).  General naming convention for tests is that
within Tfit/test/src exists a test\_filename.cpp where "filename" is an
existing file in Tfit/src.   Within this test file, individual objects
are the test set name and test names follow the method\_behavior naming 
convention.  Note that for gtest: Test(test set name, test name).  Fixture
classes (TEST\_F) are used when the setup of a test needs to be 
repeated frequently.  In these cases the class name is infoNameTest where
infoName is some informative name that includes the Class name.

Test coverage statistics and reports can be obtained through gcov
and lcov using the same Makefile (cd Tfit/test/build; cmake ..; make;
./TestTfit; make gcov; make lcov).  Note that lcov produces web parsable
coverage reports (file:///.../Tfit/test/build/lcoverage/index.html).

Furthermore, the code (both main code base Tfit/src and Unit testing 
    Tfit/test/src) have documentation following Doyxgen standards.   
The Doxygen configuration file
lives at Tfit/docs/doxyfile.in and documentation can be build (cd Tfit/docs; 
    doxygen doxyfile.in).   This manual resides within Tfit/docs/manual and 
is built using a Makefile (cd Tfit/docs/manual; make all ).  UML diagrams are
constructed using metapost (*.mp).  A User's guide has been stubbed and should
eventually be makeable via make userguide.

\section{Data Handling Design}
Tfit uses signed bedGraphs as data input.  A bedGraph has the basic format of: 
chromosome /\t start /\t stop /\t coverage.  Tfit assumes all coordinates are
reported as zero based half open coordinates, as described in the bedGraph standard.  
The coverage value is expected to be signed according to strand (e.g. positive 
numbers refer to the positive strand whereas negative numbers refer to the 
negative strand).  Thus each position in the genome may be represented in a 
signed bedGraph twice, once per sign/strand.

\includegraphics{datahandling1.mps}

Tfit attempts to fit one or more instances of the RNA polymerase model to a 
region of interest.  If the only input is the signed bedGraph, the regions of 
interest are inferred directly from the bedGraph (chromosome identifiers 
and coordinates).  Alternatively (and the preferred method), the user can 
provide a bed file (bed3, bed4, bed6, or bed12) to specify the regions of 
interest.  Each line of a bed file is stored as the corresponding gInterval.
When a user provides a ROI bed file, Tfit will *only* consider data within 
these regions (e.g. gIntervals correspond to bed entries and data outside these 
regions is ignored).  Generally, bedGraphs are read in as PointCov (position, 
coverage) in a strand specific fashion within the RawData object.  Notably,
the bed12 format is adapted here to provide Tfit with seed information.  [See
the Seeds object and information for more details.]

As most genomes contain multiple ROI (or multiple chromosomes), the contents
are stored within an indexed collection called SetROI.  Each chromosome name 
(column 1 of bedGraph and/or bed files) is converted into a numerical index.
The bimap converts between the numerical index (int) and chromosome name (string).
For each numerical index, a set of gIntervals is stored (is this ordered/sorted?). 
For rapid searching, a searchable centered index tree (CITree) can also be 
constructed for each numerical index.  The CITree can be destroyed to save 
memory.  

Once the full bedGraph is read, each gInterval's RawData is the converted into
a data Interval (dInterval), in which the data is coordinate transformed (all
dIntervals begin at position zero), binned and scaled (for numerical
stability).  The RawData can be retained (unadulterated input data) or removed 
from memory (destroyed to reduce memory footprint).  The dInterval can 
process coordinate transformations (between data coordinates, genomic 
coordinates, and indexed values).  It is also the fundamental data unit 
used by the EM algorithm for each interval.

Seeds: It is worth noting that bed12 data is interpreted in a very specific way
in the Tfit framework.   Exon lengths are weights and exon starts are 
positions for possible seeds to the EM algorithm (see EM algorithm 
section for more information).

\clearpage
\section{Models Design}
The details of the model were originally published in Azofeifa 2017 and
Azofeifa 2018.  We will repeat and augment the model description here for
completeness.  

\includegraphics{models1.mps}

Currently there are multiple models available:

\begin{itemize}
\item The Full Model (as described in Azofeifa 2017)
\item The Full Model (as described in Azofeifa 2018), e.g. includes
the footprint parameter
\item The Bidirectionals only model
\item The LIET model (not yet available, planned)
\end{itemize}

\section{EM Algorithm}
The details of model inference by the expectation maximization algorithm was
originally described in Azofeifa 2017.  We will repeat and augment the 
algorithm description here for completeness.

\includegraphics{EMalg1.mps}

\section{More on Seeding the EM algorithm}
Performance of the EM algorithm is strongly dependent on either having 
many iterations (to find an optimal parameter regime) or strong priors
(i.e. high quality seeds).  There are multiple seeding options:

\begin{itemize}
\item Random seeds
\item User defined seeds (provided by bed12)
\item Template matching (originally described in Azofeifa 2018)
\end{itemize}

\section{Customizing the algorithm}
This is about fine tuning algorithm control, model constraints, and 
the wealth of options available.   This section should also describe 
how Tfit does I/O of parameters (argparm interface, JSON) and how
to maximize reproducibility. 

\section{Leveraging Compute Resources}
MPI and Threading -- playing nice with compute clusters.

\small{
  \bibliographystyle{abbrv}
  \bibliography{rddowell}{}
}
\end{document}

